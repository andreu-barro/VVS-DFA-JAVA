%%%%%%%%%%%%%%%%%%%%%%%%%%%%%%%%%%%%%%%%%%%%%%%%%%%%%%%%%%%%%%%%%%%%%%%
% This document is based on the template: Large Colored Title Article %
%                                         Version 1.1 (25/11/12)      %
%                                                                     %
% The template was downloaded from: http://www.LaTeXTemplates.com     %
%                                                                     %
% Original author:                                                    %
% Frits Wenneker (http://www.howtotex.com)                            %
%                                                                     %
% License:                                                            %
% CC BY-NC-SA 3.0 (http://creativecommons.org/licenses/by-nc-sa/3.0/) %
%                                                                     %
% Author of this version:                                             %
% Laura M. Castro (http://www.madsgroup.org/staff/laura)              %
%                                                                     %
% Original licensing terms are maintained                             %
%%%%%%%%%%%%%%%%%%%%%%%%%%%%%%%%%%%%%%%%%%%%%%%%%%%%%%%%%%%%%%%%%%%%%%%

%----------------------------------------------------------------------------------------
%	PACKAGES AND OTHER DOCUMENT CONFIGURATIONS
%----------------------------------------------------------------------------------------

\documentclass[DIV=calc,paper=a4,fontsize=11pt,onecolumn]{scrartcl} % A4 paper and 11pt font size

\usepackage[a4paper,margin=3cm]{geometry} % 2cm margins

\usepackage[galician]{babel} % Galician language/hyphenation
\usepackage[utf8]{inputenc}
\usepackage[protrusion=true,expansion=true]{microtype} % Better typography
\usepackage{amsmath,amsfonts,amsthm} % Math packages
\usepackage[svgnames]{xcolor} % Enabling colors by their 'svgnames'
\usepackage[hang,small,labelfont=bf,up,textfont=it,up]{caption} % Custom captions under/above floats in tables or figures
\usepackage{booktabs} % Horizontal rules in tables
\usepackage{fix-cm}   % Custom font sizes - used for the initial letter in the document

\usepackage{sectsty}  % Enables custom section titles
\allsectionsfont{\usefont{OT1}{phv}{b}{n}} % Change the font of all section commands

\usepackage{fancyhdr} % Needed to define custom headers/footers
\pagestyle{fancy}     % Enables the custom headers/footers
\usepackage{lastpage} % Used to determine the number of pages in the document (for "Page X of Total")

% Headers - all currently empty
\lhead{}
\chead{}
\rhead{}

% Footers
\lfoot{\textsc{vvs-informe-prácticas}}
\cfoot{}
\rfoot{\footnotesize Páxina \thepage\ de \pageref{LastPage}} % "Page 1 of 2"

\renewcommand{\headrulewidth}{0.0pt} % No header rule
\renewcommand{\footrulewidth}{0.4pt} % Thin footer rule

\definecolor{UDC}{RGB}{206,0,124}
\definecolor{DarkUDC}{rgb}{0.75,0.75,0.75}
\definecolor{LightUDC}{RGB}{128,128,128}

\usepackage{lettrine} % Package to accentuate the first letter of the text
\newcommand{\initial}[1]{ % Defines the command and style for the first letter
\lettrine[lines=3,lhang=0.3,nindent=0em]{
\color{UDC}
{\textsf{#1}}}{}}

%----------------------------------------------------------------------------------------
%	TITLE SECTION
%----------------------------------------------------------------------------------------

\usepackage{titling} % Allows custom title configuration

\newcommand{\HorRule}{\color{UDC} \rule{\linewidth}{1pt}} % Defines the pink horizontal rule around the title

\pretitle{\vspace{-30pt} \begin{flushleft} \HorRule \fontsize{20}{20} \usefont{OT1}{phv}{b}{n} \color{DarkUDC} \selectfont} % Horizontal rule before the title

\title{INFORME DE PRÁCTICAS} % Your article title

\posttitle{\par\end{flushleft}\vskip 0.5em} % Whitespace under the title

\preauthor{\begin{flushleft}\large \lineskip 0.5em \usefont{OT1}{phv}{b}{sl} \color{DarkUDC}} % Author font configuration

\author{Repositorio de proxecto: URL \\
        Participantes no proxecto: ESTUDANTES}

\postauthor{\footnotesize \usefont{OT1}{phv}{m}{sl} \color{Black} % Configuration for the institution name
\par\end{flushleft}\HorRule} % Horizontal rule after the title

\date{\sffamily Validación e Verificación de Software} % Add a date here if you would like one to appear underneath the title block

%----------------------------------------------------------------------------------------

\usepackage{graphicx}
\usepackage{hyperref}
\hypersetup{colorlinks=true,
            allcolors=UDC}

\usepackage{array}
\usepackage{colortbl}

%----------------------------------------------------------------------------------------

\newcommand{\hint}[1]{\begin{quote}\itshape #1 \end{quote}}

%----------------------------------------------------------------------------------------

\begin{document}

\maketitle % Print the title
\thispagestyle{fancy} % Enabling the custom headers/footers for the first page 
\clearpage

%----------------------------------------------------------------------------------------
%	ARTICLE CONTENTS
%----------------------------------------------------------------------------------------

\section{Descrición do proxecto}

\hint{Breve descrición do proxecto software.}

\section{Estado actual}

\hint{Listaxe de funcionalidades, as súas especificacións, as persoas
  responsables do seu desenvolvemento, e as persoas responsables do proceso de
  proba.}

\subsection{Compoñentes avaliados}

\hint{Para cada compoñente: funcionalidades nas que participa, número
  de probas obxectivo, número de probas preparadas, porcentaxe
  executada e porcentaxe superada. Se esta información é profusa e se
  almacena noutra fonte, referencia á fonte. Se é cambiante,
  referencia a unha \emph{shapshot} ou resumo do mais destacado.}

\section{Especificación de probas}

\hint{Aquí se listarán as probas a realizar para cada compoñente
  (descrición de escenarios, entradas utilizadas e saídas desexadas,
  ferramentas de proba utilizadas, criterio de
  éxito/suspensión/abandono\dots). }

\section{Rexisto de probas}

\hint{Notas relevantes sobre o proceso de probas realizado até o momento,
  especialmente causas de atrasos, etc. Se esta información é profusa e se
  almacena noutra fonte, referencia á fonte. Se é cambiante, referencia a unha
  \emph{shapshot} ou resumo do mais destacado.}

\section{Rexistro de erros}

\hint{Notas relevantes sobre as probas realizadas que revelaron incumprimentos
  das especificacións. Se esta información é profusa e se almacena noutra fonte,
  referencia á fonte. Se é cambiante, referencia a unha \emph{shapshot} ou
  resumo do mais destacado.}

\section{Estatísticas}

\hint{Deben incluírse como mínimo:
  \begin{itemize}
    \item Número de erros encontrados diariamente e semanalmente.
    \item Nivel de progreso na execución das probas.
    \item Análise do perfil de detección de erros (lugares, compoñentes, tipoloxía).
    \item Informe de erros abertos e pechados por nivel de criticidade.
    \item Avaliación global do estado de calidade e estabilidade actuais.
  \end{itemize}}

\section{Outros aspectos de interese}

\hint{Neste apartado se incluirán todos aqueles aspectos e detalles que non se
  mencionaran nos puntos anteriores, pero que o equipo do proxecto considere que
  poden aportar valor.}

\end{document}
